\documentclass[letterpaper,spanish,10pt]{article}
\usepackage[latin1]{inputenc}    % Agregar y acentos
\usepackage{babel}               % Soporte multilenguajes
\usepackage{avant}               % Tipo de fuente
%\usepackage{fancyheadings}       %% Topes y pies de pï¿œina
\usepackage[dvips]{graphicx}     % Inclusion de imagenes .eps
\usepackage{url}                 % Agregar Links soporte de ~
\usepackage{verbatim}
\usepackage{geometry}
\usepackage{url}
\usepackage{amsfonts}
\usepackage{amssymb}
%\usepackage{txfonts}
%\usepackage{emphoff}
%\usepackage{pxfonts}
%\usepackage{fancybox}
\usepackage{latexsym}
%\usepackage{fancyvrb}
\usepackage{graphicx}
%\usepackage{wasysym}
%\renewcommand{\baselinestretch}{1.5}
\parskip=7mm
\pagestyle{myheadings}
\geometry{tmargin=2.5cm, bmargin=2.5cm, lmargin=2.5cm, rmargin=2.5cm}
\markright{\hrulefill Proyecto Hevelius $\; \;$}


%opening
\title{{\Huge \bf Proyecto Hevelius} \\ {\Large Empresa DevNull} \\ {\small Riesgos}}

\author{
{\bf Carlos Guajardo Miranda} \\ Jefe de Proyecto \\ \url{cguajard@alumnos.inf.utfsm.cl} \\ cel. 09-95046118 
\and
{\bf Marina Pilar Daza} \\ Miembro del Equipo \\ \url{mpilar@alumnos.inf.utfsm.cl} \\ cel. 09-84085407
\and
{\bf Esteban Espinoza Mart\'inez} \\ Miembro del Equipo \\ \url{eespinoz@alumnos.inf.utfsm.cl} \\ cel. 09-85596939
\and
{\bf Tom\'as Staig Fern\'andez} \\ Miembro del Equipo \\ \url{tstaig@alumnos.inf.utfsm.cl} \\ cel. 09-97615666
}


\date{13 de abril de 2007}


\begin{document}

\maketitle
%\tableofcontents


%\input{parte1}
%\input{parte2}
\newpage

\section{Identificar y listar en orden de prioridad los 3 riesgos principales asociados al desarrollo del producto (carácter técnico).}

\begin{itemize}
	\item No ser modular
	\item No poder convertir de manera correcta las coordenadas R.A.Dec. en hora sideral a Altitud/Azimuth.
	\item No poseer interfaz amigable
\end{itemize}


\section{Seleccionar y justificar el riesgo principal a abordar.}

Riesgo Principal: \bf{No poder convertir de manera correcta las coordenadas R.A.Dec. en hora sideral a Altitud/Azimuth.}\\*







\end{document}
