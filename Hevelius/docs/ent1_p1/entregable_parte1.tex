\documentclass[letterpaper,spanish,10pt]{article}
\usepackage[latin1]{inputenc}    % Agregar y acentos
\usepackage{babel}               % Soporte multilenguajes
\usepackage{avant}               % Tipo de fuente
%\usepackage{fancyheadings}       %% Topes y pies de p'agina
\usepackage[dvips]{graphicx}     % Inclusion de imagenes .eps
\usepackage{url}                 % Agregar Links soporte de ~
\usepackage{verbatim}
\usepackage{geometry}
\usepackage{url}
\usepackage{amsfonts}
\usepackage{amssymb}
%\usepackage{txfonts}
%\usepackage{emphoff}
%\usepackage{pxfonts}
%\usepackage{fancybox}
\usepackage{latexsym}
%\usepackage{fancyvrb}
\usepackage{graphicx}
%\usepackage{wasysym}
%\renewcommand{\baselinestretch}{1.5}
\parskip=7mm
\pagestyle{myheadings}
\geometry{tmargin=2.5cm, bmargin=2.5cm, lmargin=2.5cm, rmargin=2.5cm}
\markright{\hrulefill Proyecto Hevelius $\; \;$}


%opening
\title{{\Huge \bf Proyecto Hevelius} \\ {\Large Empresa DevNull} \\ {\small Riesgos}}

\author{
{\bf Carlos Guajardo Miranda} \\ Jefe de Proyecto \\ \url{cguajard@alumnos.inf.utfsm.cl} \\ cel. 09-95046118 
\and
{\bf Marina Pilar Daza} \\ Miembro del Equipo \\ \url{mpilar@alumnos.inf.utfsm.cl} \\ cel. 09-84085407
\and
{\bf Esteban Espinoza Mart\'inez} \\ Miembro del Equipo \\ \url{eespinoz@alumnos.inf.utfsm.cl} \\ cel. 09-85596939
\and
{\bf Tom\'as Staig Fern\'andez} \\ Miembro del Equipo \\ \url{tstaig@alumnos.inf.utfsm.cl} \\ cel. 09-97615666
}


\date{20 de abril de 2007}


\begin{document}

\maketitle
%\tableofcontents


%\input{parte1}
%\input{parte2}
\newpage

\section{Identificar y listar en orden de prioridad los 3 riesgos principales asociados al desarrollo del producto (carácter técnico).}

\begin{itemize}
	\item Convertir de manera correcta las coordenadas R.A.Dec. en hora sideral a Coordenadas Horizontales.
	\item Permitir uso de otro sistema de pointing (modularidad).
	\item Interfaz amigable para el uso de astr\'onomos
\end{itemize}


\section{Seleccionar y justificar el riesgo principal a abordar.}

Riesgo Principal: \textbf{Convertir de manera correcta las coordenadas R.A.Dec. en hora sideral a Coordenadas Horizontales.}\\*

Cuando un astr\'onomo quiere realizar observaciones tiene que tener presente el lugar donde se ubica el objeto en cuesti''on.
Se sabe que los astr\'onomos poseen una definici\'on de tiempo diferente a la que se usa normalmente, llamada \textit{hora sideral},
la cual se define como la medida del tiempo basada en el movimiento de la Tierra respecto de las estrellas (el tiempo entre 2 
pasos consecutivos de una estrella cualquiera por el meridiano del lugar nos definen el d\'ia sideral); aqu\'i se produce un 
ligero desfase diario respecto al d\'ia solar, debido al movimiento de traslaci\'on de la Tierra, superando \'este en 3 minutos 
y 56 segundos al d\'ia sideral.

Al poseer esta hora y el ángulo de ubicaci\'on del objeto respecto al plano ecuatorial, un astr\'onomo sabe hacia donde debe mirar. 
Sin embargo, un telescopio y su software ocupan el Sistema de Coordenadas Horizontales, el cual para determinar la posici\'on de 
un objeto, un observador deber\'a medir su altura que es la distancia angular desde el horizonte hasta la estrella. En segundo lugar, 
tendrá que determinar el \'angulo que forma la estrella con una direcci\'on que se toma como origen, generalmente el sur, medida 
sobre el horizonte y en sentido horario.

El riesgo de que Hevelius transforme de manera err\'onea las coordenadas es latente e implicar\'ia que el telescopio no observe
lo que se espera, por lo que se perder\'ia tiempo de observaci\'on en corregir los errores de software (valioso recurso para los
astr\'onomos), se apuntar\'ia a lugares potencialmente peligrosos para el lente del telescopio y har\'ia que Hevelius fuera un software
inutilizable en los observatorios.



\section{Identificar medidas de mitigaci\'on para el riesgo principal (indicar qu\'al artefactos se construir\'on para afrontar el riesgo).}

\begin{itemize}
	\item Modelo de caso de usos de la tranformaci\'on de las coordenadas.
	\item Investigaci\'on de las coordenadas.
	\item Contratos derivados de los casos de uso de la transformaci\'on de las coordenadas.
\end{itemize}


\section{Explicar/justificar c\'omo las medidas mitigar\'an el riesgo principal identificado.}

\subsection{Modelo de Casos de Uso}

Al realizar un modelo de casos de uso de la acci\'on, podemos identificar de mejor manera los problemas que podr\'ian ocurrir en el c\'alculo de las coordenadas, ya que separan los pasos de la acci\'on. De esta manera, podemos ver y separar los problemas que pueden ocurrir, ya sea por ingreso de datos err\'oneos, error en los c\'alculos mismos, en la salida de los datos, entre otros. Y, de esta manera, asegurar que el telescopio no reciba coordenadas que inesperadas.

\subsection{Investigaci\'on de las Coordenadas}

Es necesario investigar sobre las coordenadas, tanto de Altitud/Azimuth como R.A.Dec. y hora sideral. Adem\'as, se debe investigar sobre la relaci\'on de ambas coordenadas de manera de aprender c\'omo se realiza la transformaci\'on entre ellas y para poder asegurar que los c\'alculos que se har\'an, se hagan de forma correcta.

\subsection{Contratos derivados de los casos de uso de la transformaci\'on de las coordenadas.}

Al utilizar los contratos asociados a los casos de uso, podemos ver en m\'as detalle lo que sucede en cada uno de ellos, qu\'e es lo que necesitan para funcionar, las precondiciones y datos de entrada, y lo que tiene que entregar cada uno, la salida y las postcondiciones. Al tener esta informaci\'on podemos desglosar de mejor manera los problemas, para revisar cada posible error de manera minuciosa.


\end{document}
