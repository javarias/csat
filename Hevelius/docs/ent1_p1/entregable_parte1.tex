\documentclass[letterpaper,spanish,10pt]{article}
\usepackage[latin1]{inputenc}    % Agregar y acentos
\usepackage{babel}               % Soporte multilenguajes
\usepackage{avant}               % Tipo de fuente
%\usepackage{fancyheadings}       %% Topes y pies de pï¿œina
\usepackage[dvips]{graphicx}     % Inclusion de imagenes .eps
\usepackage{url}                 % Agregar Links soporte de ~
\usepackage{verbatim}
\usepackage{geometry}
\usepackage{url}
\usepackage{amsfonts}
\usepackage{amssymb}
%\usepackage{txfonts}
%\usepackage{emphoff}
%\usepackage{pxfonts}
%\usepackage{fancybox}
\usepackage{latexsym}
%\usepackage{fancyvrb}
\usepackage{graphicx}
%\usepackage{wasysym}
%\renewcommand{\baselinestretch}{1.5}
\parskip=7mm
\pagestyle{myheadings}
\geometry{tmargin=2.5cm, bmargin=2.5cm, lmargin=2.5cm, rmargin=2.5cm}
\markright{\hrulefill Proyecto Hevelius $\; \;$}


%opening
\title{{\Huge \bf Proyecto Hevelius} \\ {\Large Empresa DevNull} \\ {\small Riesgos}}

\author{
{\bf Carlos Guajardo Miranda} \\ Jefe de Proyecto \\ \url{cguajard@alumnos.inf.utfsm.cl} \\ cel. 09-95046118 
\and
{\bf Marina Pilar Daza} \\ Miembro del Equipo \\ \url{mpilar@alumnos.inf.utfsm.cl} \\ cel. 09-84085407
\and
{\bf Esteban Espinoza Mart\'inez} \\ Miembro del Equipo \\ \url{eespinoz@alumnos.inf.utfsm.cl} \\ cel. 09-85596939
\and
{\bf Tom\'as Staig Fern\'andez} \\ Miembro del Equipo \\ \url{tstaig@alumnos.inf.utfsm.cl} \\ cel. 09-97615666
}


\date{20 de abril de 2007}


\begin{document}

\maketitle
%\tableofcontents


%\input{parte1}
%\input{parte2}
\newpage

\section{Identificar y listar en orden de prioridad los 3 riesgos principales asociados al desarrollo del producto (carácter técnico).}

\begin{itemize}
	\item No ser modular
	\item No poder convertir de manera correcta las coordenadas R.A.Dec. en hora sideral a Altitud/Azimuth.
	\item No poseer interfaz amigable
\end{itemize}


\section{Seleccionar y justificar el riesgo principal a abordar.}

Riesgo Principal: \bf{No poder convertir de manera correcta las coordenadas R.A.Dec. en hora sideral a Altitud/Azimuth.}\\*

\[explicacion extendida del riesgo\]

Como ya hemos mencionado en trabajos anteriores, Hevelius se basa en el mundo de la astronom\'ia. Es por esto que debemos preocuparnos por los sistemas de coordenadas que ellos usan, tal como mencionamos en nuestro informe de requerimientos, la conversi\'on de coordenadas R.A.Dec. en horas siderales a Altitud/Azimuth, relacionada a la ubicaci\'on de los objetos de observaci\'on es de gran importancia. Esto porque una falla en este c\'alculo podr\'ia traer consecuencias graves, como la p\'erdida de horas de observaci\'on, que son bastante caras, sin contar que retrasa la investigaci\'on que se est\'e realizando, adem\'as, el equipo corre un gran riesgo, ya que al moverse a alguna coordenada err\'onea puede da\~nar

Si pasamos mal las coordenadas de R.A.Dec. en hora sideral a Altitud/Azimuth de la ubicación del objeto a observar, no sólo impide la investigación sino que se pierde tiempo de observación e aumenta el riesgos de posibles daños al telescopio.




\section{Identificar medidas de mitigación para el riesgo principal (indicar qué artefactos se construirán para afrontar el riesgo).}
\begin{itemize}
	\item 
\end{itemize}


\section{Explicar/justificar cómo las medidas mitigarán el riesgo principal identificado.}



\end{document}
