\documentclass[letterpaper,spanish,10pt]{article}
\usepackage[latin1]{inputenc}    % Agregar y acentos
\usepackage{babel}               % Soporte multilenguajes
\usepackage{avant}               % Tipo de fuente
%\usepackage{fancyheadings}      % Topes y pies de p'agina
\usepackage[dvips]{graphicx}     % Inclusion de imagenes .eps
\usepackage{url}                 % Agregar Links soporte de ~
\usepackage{verbatim}
\usepackage{geometry}
\usepackage{url}
\usepackage{amsfonts}
\usepackage{amssymb}
%\usepackage{txfonts}
%\usepackage{emphoff}
%\usepackage{pxfonts}
%\usepackage{fancybox}
\usepackage{latexsym}
%\usepackage{fancyvrb}
\usepackage{graphicx}
%\usepackage{wasysym}
%\renewcommand{\baselinestretch}{1.5}
\parskip=7mm
\pagestyle{myheadings}
\geometry{tmargin=4cm, bmargin=4cm, lmargin=2.5cm, rmargin=2.5cm}
\markright{\hrulefill Proyecto Hevelius $\; \;$}


%opening
\title{{\Huge \bf Proyecto Hevelius} \\ {\Large Empresa DevNull} \\ {\small Plan de Proyecto}}

\author{
{\bf Carlos Guajardo Miranda} \\ Jefe de Proyecto \\ \url{cguajard@alumnos.inf.utfsm.cl} \\ cel. 09-95046118 
\and
{\bf Marina Pilar Daza} \\ Miembro del Equipo \\ \url{mpilar@alumnos.inf.utfsm.cl} \\ cel. 09-84085407
\and
{\bf Esteban Espinoza Mart\'inez} \\ Miembro del Equipo \\ \url{eespinoz@alumnos.inf.utfsm.cl} \\ cel. 09-85596939
\and
{\bf Tom\'as Staig Fern\'andez} \\ Miembro del Equipo \\ \url{tstaig@alumnos.inf.utfsm.cl} \\ cel. 09-97615666
}


\date{25 de mayo de 2007}


\begin{document}

% Portada
\maketitle
\newpage

% Indices
\tableofcontents{}
\newpage




\section{Introducci\'on}
En el presente documento se da a conocer el plan del Proyecto Hevelius, el cual 
tiene por objetivo mostrar el estudio realizado por la Empresa DevNull. 
En este estudio se contemplan las soluciones al desaf\'io planteado por el grupo
ACS-UTFSM, as\'i como la concretitud de los requerimientos de \'estos.

Se advierte que el car\'acter t\'ecnico, desarrollado en algunos items del documento,
est\'a dirigido a discusiones concretas y son comprensibles por el grupo ACS-UTFSM 
y por personas vinculadas con el tema.

El documento se estructura de la siguiente forma:
\begin{itemize}
        \item \textbf{Soluci\'on conceptual:} En la cual se describe el problema 
actual, se bosquejan posibles soluciones y, finalmente, se escoge la mejor alternativa.
        \item \textbf{T\'ecnicas y herramientas de desarrollo:} Esto es, definir 
los elementos t\'ecnicos con que se construir\'a la soluci\'on y la plantilla de trabajo.
        \item \textbf{Gesti\'on de riesgos:} En esta secci\'on se identificar\'an, 
clasificar\'an y se propondr\'an estrategias de mitigaci\'on y contingencia para los 
peligros ocurrentes del proyecto.
        \item \textbf{Implementaci\'on:} En la cual se explica la incorporaci\'on del 
nuevo sistema en las instalaciones del cliente.
        \item \textbf{Planificaci\'on de actividades:} Esto significa describir el 
proceso que se seguir\'a para llevar a cabo la soluci\'on propuesta.
\end{itemize}

En el contexto m\'as general, el desaf\'io planteado por el grupo ACS-UTFSM, es 
crear un sistema de control de telescipios capaz de poder manejar cualquier telescopio 
que se conecte a trav\'es de las diferentes coordenadas utilizadas en el mundo 
astron\'omico.\\

Lo que se espera crear consiste en una interfaz gr\'afica que permita operar al 
alg\'un telescopio de manera remota, lograr un control en tiempo real y generar 
registros para posteriores an\'alisis de los datos recibidos por el telescopio.\\

La mejor soluci\'on ideada, es el dise\~no y construcci\'on de un producto de software dise\~nado 
para solventar los problemas actuales y cumplir con los requerimientos del cliente.\\

Los riesgos, que se detallan en el cap\'itulo 4, corresponden a los peligros identificados 
que pueden aparecer durante el desarrollo del proyecto, entre ellos se destaca: la poca 
escalabilidad del sistema de control y el no cumplimiento de los est\'andares ALMA.




\newpage
\section{Soluci\'on Conceptual} %%% TOMAS
\subsection{Diagn\'ostico de la situaci\'on actual}
\subsubsection{Situaci\'on Actual}
En la actualidad cada telescopio 



\subsubsection{Identificaci\'on de problemas y deficiencias}
\textbf{Unicidad de Software:} En la actualidad existen diversos tipos de telescopios, 
los cuales est\'an implementados de manera diferente dependiendo de su dise\~nador o 
de d\'onde fueron creados. Junto con esto, aparece el problema de que cada telescopio 
posee una aplicaci\'on diferente para su control, lo que obliga a los astr\'onomos, 
operadores de telescopios y aficionados a utilizar gran parte de su tiempo aprendiendo 
a ocupar los distintos softwares para cada uno de los equipos con los que van a trabajar.

\textbf{Control:} Actualmente el control de los telescopios se debe hacer de forma local, 
es decir, los operadores de telescopios y astr\'onomos deben estar en el observatorio 
para realizar sus investigaciones, pudiendo hacerse \'este de forma remota, mejorando 
la situaci\'on para los astr\'onomos, especialmente para los que se encuentran lejos 
de los sitios de observaci\'on.

\textbf{Dificultad de Uso:} Muchos de los programas utilizados actualmente para control 
de telescopios son bastante complicados de usar, obligando a gastar una considerable 
cantidad de tiempo aprendiendo a usarlos y, tambi\'en, a usarlos frecuentemente para no 
olvidar c\'omo es que se hace.

\textbf{Seguridad del telescopio:} Es importante que el telescopio tenga medios para 
protegerse de los distintos eventos que puedan ocurrir: luminosidad alta, clima inadecuado, 
entre otros. 

\newpage

\subsection{Caracterizaci\'on del cambio}
\subsubsection{Caracter\'icticas y Potencialidades deseadas}


\paragraph{Caracter\'isticas espec\'ificas deseadas para el producto.}

	\begin{itemize}
	\item \textbf{Control por internet de telescopios:} Se quiere que el sistema 
pueda funcionar situado en cualquier parte del mundo permitiendo controlar alg\'un 
telescopio que se encuentre en otro lugar geogr\'afico.

	\item \textbf{Interfaz Gr\'afica:} El software de control de telescopios debe 
tener una interfaz agradable a los usuarios y permitir el acceso eficiente a las 
funcionalidades que se requieran, adem\'as, debe mostrar siempre en pantalla la 
informaci\'on de mayor importancia.

	\item \textbf{Reproducci\'on de lo que ve la c\'amara:} El sistema debe 
mostrar a donde apunta el telescopio en todo momento de observaci\'on, por medio de 
la c\'amara CCD.

	\item \textbf{Interacci\'on con ACS:} Es necesario que el sistema interact\'ue con 
los telescopios por medio de ACS, de manera que \'este sea el que se conecte 
directamente con los observatorios y telescopios.

	\item \textbf{Ajustar posici\'on del telescopio bajo sistema de coordenadas 
ecuatoriales:} El sistema debe poder recibir las coordenadas que se quiere observar 
y convertirlas a las coordenadas que utiliza el telescopio para poder moverlo a esa 
direcci\'on.

	\item \textbf{Mover el telescopio a la hora sideral:} El sistema debe tener la 
funcionalidad de seguir la posici\'on que se est\'a observando, ya que si no se 
hace, parecer\'ia que lo observado se ve desplazando.

	\item \textbf{Impedir observaciones a lugares con luminosidad lunar:} El sistema 
debe evitar que el telescopio apunte a direcciones con notoria luminosidad lunar, 
debido a que esta luminosidad puede da\~nar severamente los lentes del telescopio.

	\item \textbf{Mostrar modelo visual del telescopio:} Debido a que el telescopio se 
quiere manipular de forma remota, es necesario otorgar alguna forma que permita 
ver a la persona que lo est\'e operando, en qu\'e estado se encuentra. Para esto, 
el sistema debe tener un modelo visual que se comporte de la misma forma que lo 
hace el telescopio real.

	\item \textbf{Ajuste manual del telescopio:} El sistema debe permitir controlar el 
telescopio manualmente para permitir ajustes menores, que ayuden a corregir errores 
en la direcci\'on que se observa, que pudieran ocurrir por factores externos, como 
es la deflexi\'on por el peso propio del telescopio en algunas posiciones.

	\item \textbf{Detener de forma inmediata el telescopio en caso de emergencia:} El 
sistema tiene que tener una opci\'on de emergencia para detener el telescopio de 
forma inmediata para evitar cualquier da\~no que se crea que pueda ocurrir. Por 
ejemplo, da\~no por alguna variaci\'on en las condiciones clim\'aticas.

	\item \textbf{Controlar acceso a la aplicaci\'on (Sesiones):} El sistema debe tener 
acceso para los distintos usuarios, de manera que cada uno tenga su propia 
estad\'istica de lo observado.

	\item \textbf{Guardar coordenadas de observaci\'on realizadas:} El sistema debe 
guardar registro de las coordenadas observadas por cada usuario del sistema. De 
esta forma ayuda a que se puedan repetir observaciones y a realizar estudios sobre 
\'estas.

	\end{itemize}


\newpage

\paragraph{Relaci\'on de las caracter\'isticas con los problemas identificados.}

	\begin{itemize}

	\item El control por internet va a ayudar a solucionar el problema de tener 
que estar en el lugar de observaci\'on al momento de controlar al telescopio.

	\item La interfaz gr\'afica va a ayudar a disminuir la dificultad de uso, 
ocultando informaci\'on que no sea requerida en todo momento, pero permitiendo 
verla de manera sencilla e intuitiva.

	\item La reproducci\'on de lo que est\'a viendo el telescopio es de gran 
utilidad para la experiencia remota, debido a que sino hiciera esto, no se podr\'ia 
ver lo que est\'a viendo el telescopio, hasta que se enviara alg\'un informe a 
quien controlaba el telescopio.

	\item La interacci\'on con ACS es una de las caracter\'isticas principales 
para el control gen\'erico de telescopios y el control de telescopios por medio 
de internet, pues es esta plataforma la que permite la comunicaci\'on con los 
telescopios en los diferentes observatorios del mundo.

	\item Mover el telescopio a la hora sideral reduce la dificultad de uso 
para el seguimiento de la observaci\'on de alg\'un objeto, puesto que nos permite 
ver en todo momento al objeto deseado, sin necesidad de realizar tareas adicionales.

	\item Al impedir que el telescopio apunte a lugares con luminosidad lunar 
se reduce la dificultad de uso, puesto que no es necesario estar preguntandose 
todo el tiempo si el telescopio va a apuntar a lugares potencialmente da\~ninos 
para el mismo. Adem\'as, aumenta la seguridad del telescopio puesto que lo proteje 
de la luz lunar, uno de los factores m\'as comunes que da\~nan al telescopio.

	\item El modelo visual soluciona un aspecto muy importante de la dificultad 
de uso para el control a trav\'es de internet, ya que con este se puede saber en 
todo momento hacia d\'onde est\'a apuntando f\'isicamente el telescopio, d\'andonos 
un apoyo gr\'afico de lo que estamos haciendo. De la misma forma, tambi\'en ayuda 
a los que operan el telescopio de forma local, aunqiue ellos podr\'ian verlo 
directamente, puede ser m\'as c\'omodo verlo en la misma pantalla que est\'an trabajando.

	\item El ajuste manual ayuda a disminuir la dificultad de uso del sistema, 
puesto que con este, no es necesario intuir una direcci\'on parecida a la que 
estamos observando de manera que se vea lo que debiera, sino que simplemente 
lo movemos manualmente hasta donde debiera estar.

	\item Al dar la posibilidad de detener manualmente al telescopio, aumentamos 
en gran medida su seguridad, puesto que mediante esta opci\'on, podemos protegerlo 
de factores que no esperabamos, como son las variaciones inesperadas en el clima. 

	\item El guardar coordenadas de observaci\'on realizadas por sesi\'on facilita 
la dificultad de uso del sistema, ya que para gente no muy experimentada en el tema, 
permite repetir las observaciones hechas otros d\'ias.

	\end{itemize}

\subsubsection{Restricciones}

\begin{itemize}

\item \textbf{Econ\'omicas:} El software no presenta restricciones econ\'omicas, 
puesto que tanto el sistema operativo, como las herramientas de desarrollo que se 
van a utilizar, son gratuitas. Por otro lado, los componentes de hardware como son 
el telescopio para pruebas y la c\'amara CCD si tienen un costo, pero en este caso 
ser\'an facilitados por el cliente. Es por esto, que no vemos restricciones 
econ\'omicas peligrosas.

\item \textbf{Sociales y Culturales:} Los usuarios actuales de los programas que 
controlan telescopios han tenido que usar diferentes aplicaciones para distintos 
telescopios a lo largo del tiempo que han dedicado a esto, prefiriendo quiz\'as, 
el que usan actualmente, ya sea por costumbre o por gusto personal. Esto puede 
dificultar que se acostumbren a usar el sistema propuesto, pero se espera que el 
sistema final sea intuitivo y amigable, de manera que esto no debiera suceder.

\item \textbf{Tecnol\'ogicas:} En el aspecto tecnol\'ogico es importante destacar 
que las pruebas iniciales no necesariamente se har\'an con un telescopio de 
observatorio, en estos casos se utilizar\'a para las pruebas telescopios para 
aficionados, puesto que los costos de observaci\'on son elevados.

%%%%\item \textbf{Institucionales:}


\end{itemize}

\subsection{An\'alisis de las alternativas de soluci\'on}
\subsubsection{Alternativa 1: Desarrollo de Software basado en ACS} %%%% TOMAS
Desarrollo de un producto de software basado en la plataforma ACS que sea gen\'erico,
es decir, que nos permita controlar cualquier telescopio por medio del mismo
programa, sin la necesidad de tener un programa diferente para cada telescopio.

Se utiliza un computador como estaci\'on de trabajo de quien opere el telescopio, 
en donde todo el control se realizar\'a por medio de una interfaz gr\'afica. Este
computador requerir\'a tener acceso a internet para poder obtener componentes desde
ACS y para comunicarse con el telescopio que se quiera controlar.

La interfaz gr\'afica mostrar\'a a quien opere el telescopio el estado actual del
mismo, pudiendo verse lo que est\'a observando el telescopio por medio de la
c\'amara CCD y la disposici\'on f\'isica en que se encuentra el telescopio, por medio
del modelo hecho en OpenGL del mismo.

Es importante notar que al utilizar la plataforma ACS para la distribuci\'on de 
componentes de software, se podr\'ian usar componentes realizados por otras personas, 
asi como realizar cambios en componentes que utiliza nuestro software, obteniendo un
producto de alta modularidad, enfocado al trabajo por componentes.

\subsubsection{Alternativa 2: YYY+1} %%%% TODOS



\subsection{Soluci\'on recomendada} %%%% TOMAS
La mejor soluci\'on que encontramos es la alternativa 1, en la cual se piensa construir
un producto de software gen\'erico basado en ACS, el cual se acopla bastante bien 
con los requerimientos del cliente.

Este producto deber\'a ser modular basado en componentes con el estilo que ACS nos
impone, por lo que se tendr\'an algunos componentes sobre esta plataforma, mientras que
otros ir\'an junto con el software principal. Es importante hacer la separaci\'on
de las capas de comunicaci\'on, interfaz gr\'afica y software, para simplificar la
mantenci\'on del producto al tener las funcionalidades separadas.

Se eligi\'o esta alternativa, porque en la actualidad no hay productos que controlen
telescopios de forma gen\'erica, y esto es algo muy importante para enfocar los esfuerzos
de observaci\'on en las observaciones mismas y no en aprender a utilizar los programas
asociados al control de telescopios.


\newpage
\section{T\'ecnicas y Herramientas de desarrollo} %%% CARLOS
\subsection{Modelo de desarrollo}
Las caracter\'isticas m\'as importantes del proyecto Hevelius con
respecto a la elecci\'on de un modelo de desarrollo se expone en lo
siguiente. Primero se consideran las propiedades del producto:

\begin{itemize}
\item \textbf{Complejidad del Proyecto:} una estimaci\'on informal del proyecto
  muestra una complejidad media, la cual permite la finalizaci\'on del
  proyecto dentro del plazo predeterminado;
\item \textbf{Solidez de los Requerimientos:} c\'omo el producto a
  desarrollar est\'a inserto en un proyecto de investigaci\'on, es posible que
  los requerimientos capturados durante el an\'alisis sean completos en su generalidad,
  sujetos a pocos cambios.
\item \textbf{Innovaci\'on del Producto:} en vista de que la principal caracter\'istica
  del proyecto es la b\'usqueda de la generalidad en el control de telescopios, existen 
  errores que durante el desarrollo se descubrir\'an, lo que nesecitar\'a cambios en el
  c\'odigo implementados o en el dise\~no del software.
\end{itemize}

Adem\'as, el ambiente del desarrollo tiene las siguientes caracter\'isticas:

\begin{itemize}
\item \textbf{Tama\~no del Equipo:} cuatro personas durante la planificaci\'on y el 
  desarrollo del proyecto.
\item \textbf{Recursos Disponibles:} nuestro cliente nos provee de un lugar f\'isico, 
  computadores y un telescopio para el desarrollo de nuestro proyecto.
\item \textbf{Tratamiento de los miembros del equipo entre s\'i:} informal, ya
  que existen lazos de amistad anteriores a la formaci\'on del grupo de trabajo.
\item \textbf{Proyecci\'on en el tiempo:} se cuenta con un plazo determinado,
  debido a que el proyecto pretende ser parte de la Feria de Software 2007, 
  de aproximadamente cinco meses a partir de la fecha de entrega de este informe.
\end{itemize}

%%%%%%%%%%%%%%%%%%%%%%%%%%%%%%%%%%%%%
%%%% revizar de aqui hacia abajo %%%%
%%%%%%%%%%%%%%%%%%%%%%%%%%%%%%%%%%%%%

%%%% (ITERATIVO)



Teniendo en cuenta el conjunto de caracter\'isticas del proyecto, se
decide usar un m\'etodo \'agil de desarrollo para la
mayor\'ia de las tareas; espec\'ificamente se prevee el uso del \emph{Modelo
Evolutivo} (tambi\'en llamado \emph{dise\~no continuo} o \emph{dise\~no
incremental}) en el trabajo. Sin embargo, no es posible
usar un m\'etodo \'agil de forma exclusiva, porque es necesario cumplir con
las fechas de entregas intermedias predeterminadas por los ramos
``Ingenier\'ia de Software'' y ``Taller de desarrollo de Software''.
Tambi\'en ser\'a necesario contar al final del desarrollo con una
documentaci\'on comprensible por el cliente\footnote{que en el caso del
  proyecto presente est\'a compuesto de ingenieros inform\'aticos y
  electr\'onicos} del dise\~no y del c\'odigo del software, cosa que se
considera m\'as f\'acil usando algunos elementos de un proceso planificado.
Por
tanto, el uso parcial de un proceso planificado ser\'a inevitable.

Habiendo elegido este m\'etodo de desarrollo, es necesario realizar las
siguientes actividades durante el desarrollo para el modelo evolutivo:

\begin{enumerate}
\item \textbf{Pruebas:} los requerimientos (funcionales o no-funcionales) deben
  ser transformados en pruebas automatizables.
\item \textbf{Construir:} el c\'odigo producido debe ser siempre en un
  estado que permite su compilaci\'on.
\item \label{escuchar} \textbf{Escuchar:} La comunicaci\'on a lo largo del desarrollo del proyecto
con el cliente, debe ser fluida, esto para atender a las 
  verdaderas necesidades y para solicitar sus comentarios sobre el
  estado del desarrollo, y posibles cambios en los requerimientos.
\item \textbf{Dise\~nar:} para facilitar el proceso de codificaci\'on y permitir el
  trabajo paralelo de grupos independientes, hay que contar con un
  dise\~no del sistema.
  Esta actividad hay que hacerla durante todo el tiempo del desarrollo,
  adapt\'andose a funcionalidad agregada.
\item \label{comunicar} \textbf{Comunicaci\'on dentro del equipo:} es importante
  para el uso de un m\'etodo \'agil que la comunicaci\'on entre los integrantes funcione
  muy bien, es decir, todos deben saber la mayor\'ia del tiempo en qu\'e
  est\'an trabajando los dem\'as y cu\'ales son los cambios realizados por
  ellos.
\end{enumerate}

Para los puntos \ref{escuchar} y \ref{comunicar} se necesitan adem\'as
muchas versiones intermedias.

Para el componente planificado, hay que hacer un plan inicial de
recursos y tiempo necesitado, para lo que sirve este documento, y
controlar el progreso actual del proyecto, para lo que sirven los
\emph{fichas de estado}.
Adem\'as es necesario documentar el dise\~no actual del programa.



\subsection{Herramientas y t\'ecnicas de soporte para el desarrollo}
\subsubsection{T\'ecnicas a utilizar en el desarrollo del proyecto}

Durante el desarrollo del proyecto se utilizar\'a el m\'etodo evolutivo ya que es necesario contar con una planificaci\'on a seguir en los meses de desarrollo. Adem\'as, es necesario una documentaci\'on del desarrollo del proyecto para futuras modificaciones.\\

Se opt\'o por este m\'etodo debido a la seguridad que da su planificaci\'on al cliente, ya que \'el puede \textit{ver} en que se est\'a trabajando y cuales ser\'an los pr\'oximos pasos a seguir.\\

Dentro de los lenguajes evaluados, se opt\'o por la programaci\'on en C++ al ser este un potente lenguaje con una orientaci\'on a objetos que facilita el dise\~no gracias a las herramientas de planificaci\'on y modelamiento de UML. Todos los integrantes del grupo, cuentan con conocimientos en C y otros en C++. Para quienes no cuenta con los conocimientos en C, el paso a C++ es diminuto en comparaci\'on con otros lenguajes de programaci\'on.\\

Dentro de los lenguajes descartados, se encuentra Java. Esto se debe principalmente a que los requerimientos de hardware de su m\'aquina virtual, en muchos cosas sobrepasa las caracter\'isticas de los equipos en los que podremos desarrollar el proyecto. Esta situaci\'on a provocado que no exista una empat\'ia por parte del grupo hacia Java, debido a la ralentizaci\'on que produce en los equipos personales la realizaci\'on de otras operaciones.\\

Es posible que se utilice alguna herramienta de dise\~no vectorial para la creaci\'on de las gr\'aficas a utilizar en el software, pero a\'un no se ha descartado ni confirmado ninguna de ellas.\\

Para asegurar la calidad de software, se realizar\'an constantes revisiones del producto con el cliente, de forma que esto permita obtener una retroalimentaci\'on y as\'i mejorar o cambiar peque\~nos defectos en corto tiempo.\\

Dentro de este mismo marco, utilizaremos la herramienta de control de versiones SVN para mantener un desarrollo ordenado, una coordinaci\'on entre los desarrolladores, saber cual de los desarrolladores realiz\'o que cambio, etc. 
Adem\'as, se planea la utilizaci\'on la herramienta bugtracker\footnote{Sistema dise\~nado especialmente para manejar reportes de errores} Trac\footnote{\url{http://www.edgewall.com/trac/}}.

%%%%%%%%%%%%%%%%%%%%%%%%%%%%%%%%%%%%%%%%%%%%%%%%%%%%%%%%%%%%%%%%%%%%%%%%


\subsubsection{Herramientas o plataformas espec\'ificas a utilizar}

La plataforma operacional de Hevelius est\'a constituida por el Sistema Operativo Linux Fedora Core, debido a que es esta la distribuci\'on utilizada por nuestro cliente para desarrollar software.\\
%% AGREGAR ACS

%%%%%%%%%  REVISAR   %%%%%%%%%%%%%%%%%%%%

%%%%% VER SI ES NECESARIO UML SINO PICO

Tal como se mencionaba antes, se utilizar\'a UML como plataforma de modelamiento. De esta forma, se est\'a evaluando que herramienta CASE nos ser\'a de mayor utilidad. Entre estas est\'an Umbrello\footnote{\url{http://uml.sourceforge.net/}}, Rational Rose\footnote{\url{http://www-306.ibm.com/software/rational/}} y DBDesigner 4\footnote{\url{http://fabforce.net/dbdesigner4/}}

%%%%%%%%%%%%%%%%%%%%%%%%%%%%%%%%%%%%%%%%%%%

\subsection{Personal y capacitaci\'on del grupo de desarrollo}
\begin{itemize}

    \item Personal del equipo del proyecto.
		
La empresa DevNull cuenta actualmente con cuatro miembros, de los cuales se har\'a una breve descripci\'on de sus perfiles personales, paso primordial para conocer al equipo de trabajo:

	\begin{description}
		
	\item[Marina Alejandra Pilar Daza].

Persona con gran capacidad de liderazgo, que gusta del trabajo en equipo y por sobre todo con gran sentido de responsabilidad. 
``Echando a perder se aprende'' ayuda mucho a entender su personalidad, ya que se caracteriza por su curiosidad, la cual siempre le ha ense\~nado algo nuevo. Adem\'as posee facilidad para el aprendizaje. 

Dentro de sus \'areas de inter\'es, se encuentra todo el amplio abanico de la inform\'atica, la rob\'otica, y la biotecnolog\'ia, demostrando este inter\'es, en su activa participaci\'on en el equipo de Promoci\'on del Departamento de Inform\'atica de la UTFSM, y coordinar las Charlas T\'ecnicas del Laboratorio de Computaci\'on del mismo departamento (Labcomp).

Posee facilidad en la comunicaci\'on, adem\'as de una gran empat\'ia a la hora de comunicarse con la gente.

En el 2005, logr\'o la beca Bicentenario, y la beca Municipal de La Calera, para apoyar sus estudios por excelencia acad\'emica, dejando en manifiesto las ganas de progresar gracias a sus propios m\'eritos y esfuerzo.

Gusta de la m\'usica y el cine, tocar clarinete, jugar rol, y adem\'as, practica Nataci\'on.

Posee un dominio del Ingl\'es Avanzado (oral y escrito, t\'ecnico) y Alem\'an B\'asico (oral y escrito).

                \begin{itemize}
                \item{Conocimientos T\'ecnicos:}
                        \begin{itemize}
                        \item{Administrador de Sistemas:} Cuentas, Mail, Web, SVN, Bases de Datos.
                        \item{Sistemas Operativos:} Microsoft Windows, Linux (Fedora, Ubuntu, CentOS, Gentoo).
                        \item{Lenguajes de Programaci\'on:} C, C++, Basic, Visual Basic, Java, PHP, ASP, Javascript, Scheme, Prolog, Bash, Perl.
                        \item{Librer\'ias:} SDL, GTK, QT, CUPS, STL.
                        \end{itemize}
                \end{itemize}



	\item[Tom\'as Ignacio Staig Fern\'andez].

Persona capaz de liderar proyectos y ser part\'icipe de estos, de analizar y dise\~nar una soluci\'on para implementarla con las herramientas que el problema requiera.

Su inter\'es por el \'area de Sistemas Computacionales lo lleva a ingresar al Laboratorio de Computaci\'on (Labcomp) y ganar experiencia en la Administraci\'on de Sistemas. Esta experiencia no es s\'olo t\'ecnica, si no que humana. Trabajar con ese grupo de personas, le ha ense\~nado a aprender y escuchar las ideas del otro, y a aceptar el error cuando ha estado equivocado.

Adem\'as participa en el equipo de Promoci\'on del Departamento de Inform\'atica, donde realizan Charlas a colegios y actividades para Puertas Abiertas UTFSM. Esto lo llev\'o a ser reconocido como Alumno Destacado UTFSM los a\~nos 2004 y 2005.

La preocupaci\'on por el correcto manejo y uso del idioma espa\~nol en forma oral y escrita le ha facilitado, en muchas oportunidades, ganarse la confianza de superiores y personas de mayor rango jer\'arquico. Adem\'as del Espa\~nol, tiene la capacidad de comunicarse en Ingl\'es Coloquial y T\'ecnico Inform\'atico, oral y escrito, a nivel medio.

Sus hobbies son la lectura, el tenis, juegos de rol y por supuesto tarrear\footnote{tambi\'en conocido como ``LAN Party''. Una LAN party es un evento que reune a un grupo de personas con sus computadoras para jugar y compartir} con amigos.

                \begin{itemize}
                \item{Conocimientos T\'ecnicos:}
                        \begin{itemize}
                        \item{Administrador de Sistemas:} Cuentas, Mail, Web, SVN, Bases de Datos.
                        \item{Sistemas Operativos:} Microsoft Windows, Linux (Fedora, Ubuntu, CentOS, Gentoo).
                        \item{Lenguajes de Programaci\'on:} C, C++, Basic, Visual Basic, Java, PHP, ASP, Javascript, Scheme, Prolog, Bash, Perl.
                        \item{Librer\'ias:} SDL, GTK, QT, CUPS, STL.
                        \end{itemize}
                \end{itemize}




	\item[Esteban Ignacio Espinoza Mart\'inez].

Persona met\'odica y anal\'itica. Actualmente se encuentra finalizando su formaci\'on acad\'emica, de la cual valora sobre todo, los conocimientos t\'ecnicos que le ha entregado su estudio.  

Destaca en \'el su alto grado de responsabilidad y su facilidad de aprendizaje. Es capaz de aceptar desaf\'ios y concretarlos en el tiempo acordado, demostrando su capacidad y compromiso al trabajar en equipo e individualmente.

Se identifica con el \'area de Desarrollo de Software, aunque su inter\'es es la amplia \'area de la Computaci\'on e Inform\'atica.

Es fuertemente valorable sus conocimientos sobre variados lenguajes de programaci\'on y su habilidad para programar sobre estos, con los cuales ha llevado a la pr\'actica algunos proyectos de desarrollo personal.

Tambi\'en destacan sus habilidades deportivas en atletismo y f\'utbol. Sus hobbies son tocar arm\'onica, jugar rol y entretenerse con juegos de PC. 

Posee dominio del idioma Ingl\'es a nivel intermedio (t\'ecnico y escrito).

                \begin{itemize}
                \item{Conocimientos T\'ecnicos:}
                        \begin{itemize}
                        \item{Administrador de Sistemas:} Cuentas, Mail, Web, SVN, Bases de Datos.
                        \item{Sistemas Operativos:} Microsoft Windows, Linux (Fedora, Ubuntu, CentOS, Gentoo).
                        \item{Lenguajes de Programaci\'on:} C, C++, Basic, Visual Basic, Java, PHP, ASP, Javascript, Scheme, Prolog, Bash, Perl.
                        \item{Librer\'ias:} SDL, GTK, QT, CUPS, STL.
                        \end{itemize}
                \end{itemize}




	\item[Carlos Alberto Guajardo Miranda].

Es un alumno proveniente de la RWTH Aachen, Alemania, que este a\~no ha llegado a la UTFSM a trav\'es de una beca de intercambio obtenida en el pa\'is del cual es originario, el cual, en sus ganas de trabajar e interactuar con sus nuevos compa\~neros chilenos, eligi\'o la asignatura de Ingenier\'ia de Software, en donde se integra a la empresa Taranis, equipo el cual lo acoge desde el primer momento.

Se acaracteriza por ser una persona que siempre trata de entender las cosas a trav\'es de la comunicaci\'on con otras personas, especialmente potenciado por su necesidad de adaptaci\'on con la cultura y el idioma de Chile, y su deseo de transmitir su forma de vida en su pa\'is de procedencia. 

Es capaz de identificar problemas con rapidez y proponer soluciones, ya sean soluciones propias o buscando a la gente que m\'as adecuada para resolverlos; adem\'as, suele hacer planes factibles y realizarlos, de mostrando efectividad en el trabajo. Por lo mismo es que tambi\'en cumple con el trabajo prometido, por su consecuencia a la hora de adquirir compromisis, y conocer sus l\'imites de trabajo.

Estas caracter\'isticas personales, as\'i como su desempe\~no acad\'emico le permitieron lograr una beca del servicio alem\'an de intercambio acad\'emico (DAAD) para sustentar su estad\'ia de intercambio en la UTFSM. Sus \'areas de inter\'es en el campo inform\'atico son la fiabilidad y seguridad de sistemas, comunicaci\'on de datos, y el modelamiento matem\'atico. 

Fuera del campo inform\'atico, ha sido tenismesista por los \'ultimos 17 a\~nos de su vida. Posee dominio de los idiomas Alem\'an, el cual es su lengua materna, Ingl\'es a nivel avanzado y posee buen manejo del idioma Espa\~nol.

                \begin{itemize}
                \item{Conocimientos T\'ecnicos:}
                        \begin{itemize}
                        \item{Administrador de Sistemas:} Cuentas, Mail, Web, SVN, Bases de Datos.
                        \item{Sistemas Operativos:} Microsoft Windows, Linux (Fedora, Ubuntu, CentOS, Gentoo).
                        \item{Lenguajes de Programaci\'on:} C, C++, Basic, Visual Basic, Java, PHP, ASP, Javascript, Scheme, Prolog, Bash, Perl.
                        \item{Librer\'ias:} SDL, GTK, QT, CUPS, STL.
                        \end{itemize}
                \end{itemize}




	\end{description}


	\item Perfil del equipo ideal para el proyecto.

Para la empresa DevNull, el equipo ideal requerido para este proyecto requiere que cumpla con las siguientes caracter\'isticas:

%%%%%%%%%%%%%%%%%%%% REVISAR %%%%%%%%%%%%%%%%%%%%

		\begin{itemize}

		\item{Conocimientos de Astronom\'ia:} Esencial es el conocimiento del funcionamiento de robots, como tambi\'en conocer la l\'ogica de su funcionamiento interno de las piezas electr\'onicas y/o Hardware que posee y conocer en espec\'ifico con qu\'e tipo de robot el equipo se encuentra trabajando para el desarrollo del software.
		\item{Conocimientos sobre Sistemas:} Se requiere conocer tanto el sistema desde el dise\~no interno del robot, como su funcionamiento aut\'onomo. Conocer de su arquitectura tambi\'en es fundamental para lograr entender su funcionamiento o nuevas funcionalidades que pueda ser agregadas en el robot.
		\item{Conocimientos de Lenguajes de Programaci\'on:} 
			
			\begin{itemize}
			%%%% C++, JAVA, C , LATEX, PHYTON
			\item{C++:} En este lenguaje de programaci\'on orientado a objetos se basar\'a la funcionalidad de nuestro software.
			\item{\LaTeX:} Este lenguaje ser\'a usado para generar la documentaci\'on que surge a trav\'es de los entregables o a partir de la propia investigaci\'on que surge durante el avance del trabajo.
			\end{itemize}
		\item{Conocimientos de Software:}
		\item{Conocimientos de Sistemas Operativos:} Se requiere un conocimiento en espec\'ifico del sistema operativo Ubuntu, el cual se encuentra actualmente en la version 5.10 y pr\'oximo a cambiar a la versi\'on 6.06, debido a que es el sistema sobre el cual funciona el robot. 

		\end{itemize}

%%%%%%%%%%%%%%%%%%%%%%%%%%%%%%%%%%%%%%%%%%%%%%%%%%%%%%%
	\item Capacidad, experiencia y disponibilidad de cada miembro del equipo (disponible o buscado).

		\begin{description}

		\item[Marina Pilar]
			\begin{itemize}
                        \item{Administrador de Sistemas:} Cuentas, Mail, Web, SVN, Bases de Datos.
                        \item{Sistemas Operativos:} Microsoft Windows, Linux (Fedora, Ubuntu, CentOS, Gentoo).
                        \item{Lenguajes de Programaci\'on:} C, C++, Basic, Visual Basic, Java, PHP, ASP, Javascript, Scheme, Prolog, Bash, Perl.
                        \item{Librer\'ias:} SDL, GTK, QT, CUPS, STL.
			\item{Otros conocimientos relacionados:} Arquitectura de computadores, Redes de computadores, F\'isica (Din\'amica y Electromagnetismo), Rob\'otica a nivel b\'asico.
                        \end{itemize}
		
		\item[Tom\'as Staig]
						
			\begin{itemize}
                        \item{Administraci\'on de Sistemas:} Correo Electr\'onico, Servicios Web, Cuentas de Usuarios, Bases de Datos, DHCP, Firewall, LDAP.
                        \item{Sistemas Operativos:} Linux (Fedora Core, CentOS, Debian, Ubuntu, ArchLinux), Microsoft Windows 98/2000/Me/XP
                        \item{Lenguajes de Programaci\'on:} C, Java, Visual Basic, HTML, Javascript, ASP, PHP, SQL, Bash, Latex, Perl, Scheme, Prolog, Tck, Tk.
                        \item{Librer\'ias:} GTK, QT.
			\item{Otros conocimientos relacionados:} Arquitectura de computadores, Redes de computadores, F\'isica (Din\'amica), Rob\'otica a nivel b\'asico.
                        \end{itemize}


		\item[Esteban Espinoza]

			\begin{itemize}
                        \item{Software:} MS Office, OpenOffice, Rational Rose, SQL Server, Visual Studio 6.0, PostgreSQL, Eclipse, Klogic.
                        \item{Sistemas Operativos:} MS Windows 95, 98, XP (Nivel avanzado), RedHat Linux, Fedora Core (Nivel Medio). 
                        \item{Lenguajes de Programacion:} C, C++, Pascal, Visual Basic, ASP, PHP, SQL, JavaScript, Cobol, Dbase, Clipper, Java, Perl, Scheme, Prolog.
			\item{Otros conocimientos relacionados:} Arquitectura de computadores, Redes de computadores, F\'isica (Din\'amica y Electromagnetismo), Rob\'otica a nivel b\'asico.
                        \end{itemize}


		\item[Carlos Guajardo]

			\begin{itemize}
			\item{Sistemas Operativos:} MS Windows 95, 98, XP (Nivel Medio), RedHat Linux, Fedora Core, Ubuntu (Nivel Medio).
			\item{Lenguajes de Programacion:} C, C++, Pascal, Visual Basic, ASP, PHP, SQL, JavaScript, Java,  Prolog, entre otros.
			\item{Otros conocimientos relacionados:} Arquitectura de computadores, Redes de computadores, F\'isica (Din\'amica y Electromagnetismo), Rob\'otica a nivel b\'asico.
			\end{itemize}

		\end{description}

A partir de los datos tabulados, se puede decir que la mayor\'ia de los recursos t\'ecnicos requeridos por el equipo son cubiertos. 

        %\end{itemize}

        \item An\'alisis de insuficiencias de los miembros del equipo, si las hubiere.

%%%%%%%%%%%%%%%%%%%% REVISAR %%%%%%%%%%%%%%%%%%%%%%



A partir de lo anterior, las insufiencias que se deber\'ian cubrir son los siguientes:

		\begin{itemize}
		\item{Falta de capacitaci\'on en astronom\'ia y acs:} En conjunto los miembros del equipo s\'olo mantienen nociones b\'asicas del funcionamiento de un robot, por lo que limita el trabajo por el poco entendimiento t\'ecnico de \'este, lo que tambi\'en trae problemas de comunicaci\'on con el cliente.
		\end{itemize}

        \item Programa de mejoramiento (capacitaci\'on, mentor\'ia, incorporaci\'on de nuevos miembros...), indicando espec\'ificamente acciones a realizar y plazos.

Para el mejoramiento de las insuficiencias detectadas, el equipo recurrir\'a al siguiente plan de accion:

		\begin{itemize}
		\item{Falta de capacitaci\'on en rob\'otica:} Para mejorar esta insuficiencia, el equipo ya tom\'o su primer plan de acci\'on, que fue la asistencia a un Seminario de Rob\'otica dictado por la empresa Austral technologies (Austec), con la cual trabajan en el proyecto de Software, para conocer m\'as de cerca el mundo de la rob\'otica, con un punto a favor que al tratarse de los mismos clientes con que el equipo trabaja, el enfoque del Seminario se bas\'o en gran parte al tipo de trabajo que ellos realizan. El siguiente plan de acci\'on es asistir con el cliente en un plazo regular de una vez a la semana para recibir mayores instrucciones respecto a rob\'otica, para poder as\'i cada vez m\'as entender las propias necesidades que requieren.
		\item{Retiro de un miembro del equipo.} Para la mejora de esta insuficiencia, se proponen dos planes de acci\'on:
			\begin{itemize}
			\item{Redistribuci\'on del trabajo inicial de cinco personas en cuatro partes}
			\item{Buscar una persona externa que cumpla con el rol del integrante}
			\end{itemize}

		Para llevar a cabo alguno de estos planes, dependemos del trabajo que se prosiga con el tiempo, ya que a partir de este veremos en que temas aquel integrante a resultado especialista e imprescindible, para la b\'usqueda de un nuevo integrante durante el per\'iodo de vacaciones, luego de un an\'alisis de trabajo del integrante retirado, como tambi\'en puede darse el caso de que su trabajo sea perfectamente repartible entre los integrantes que se quedan. Como la segunda opci\'on resulta ser, a vista del equipo, la que menos compromete factores externos o que escapen de su alcance, la l\'ogica de trabajo ser\'a repartir tareas de distinto tipo a distintos integrantes del grupo, para que as\'i se especialice en varias \'areas de trabajo y as\'i evitar problemas de desconocimiento de los temas por falta de uno de los integrantes faltantes.
		\end{itemize}



%%%%%%%%%%%%%%%%%%%%%%%%%%%%%%%%%%%%%%%%%%%%%%%%%%%
\end{itemize}



\newpage
\section{Gesti\'on de Riesgos} %%% ESTEBAN
\subsection{An\'alisis de riesgos}
\begin{itemize}
\item{Riesgos de Negocio:} 
\begin{itemize}
\item{Poca escalabilidad de Hevelius con respecto a telescopios profesionales.}

Un gran riesgo es que una vez terminado el proyecto, este al intentar ser probado en un telescopio profesional no sea lo suficientemente escalable y termine por no realizar un funcionamiento adecuado.
\end{itemize}

\item{Riesgos del Proyecto:}
\begin{itemize}
\item{Ausencia de alg\'un integrante del proyecto}

La ausencia de alg\'un integrante del equipo de trabajo implicar\'ia un  mayos esfuerzo por parte del resto del equipo a dem\'as de incrementar el tiempo de desarrollo del proyecto, lo cual podr\'ia traer serios problemas.

\item{Incumplimiento con fechas.}

Todos los proyectos tienen una fecha limite para cada etapa y sus respectivas entregas. Un riesgo importante es el incumplimiento de las fechas finales, las cuales corresponden a la presentaci\'on del proyecto en La Feria de Software.	

\item{Telescopio Amateur no disponible}

Para la presentaci\'on de la feria, Hevelius ser\'a probado en un telescopio amateur el cual es propiedad de ACS-UTFSM Group, por lo que la disponibilidad de este se puede ver afectada para dicha presentaci\'on.

\item{Estimaci\'on del proyecto err\'onea}

Una mala estimaci\'on del proyecto, en cuanto a complejidad y desarrollo implicar\'ia un gran riesgo, puesto que podr\'ia darse el caso en que no se logre finalizar el proyecto dentro de la fecha estimada.

\item{Falta de conocimiento en cuanto al \'area de desarrollo}

Hevelius al tratarse de un proyecto que esta en el \'area de la astronom\'ia, obliga necesariamente al equipo de trabajo familiarizarse con t\'erminos e informaci\'on propia de dicha \'area, lo cual implica una capacitaci\'on para el equipo de manera de poder visualizar de mejor manera el problema y poder desarrollar una mejor soluci\'on, a dem\'as de permitir una mejor obtenci\'on de informaci\'on de parte del cliente. 

\item{No cumplimiento con los est\'andares del proyecto ALMA-CONICYT}

Hevelius al formar parte del proyecto ALMA-CONICYT debe cumplir ciertos est\'andares, por lo que alg\'un tipo de incumplimiento producir\'ia conflictos en cuanto a utilizaci\'on y una mala evaluaci\'on del proyecto por parte de nuestro cliente.

\end{itemize}


\item{Riesgos T\'ecnicos:}
\begin{itemize}
\item{Mala elecci\'on en cuanto a lenguajes de programaci\'on.}

Una mala elecci\'on del lenguaje de programaci\'on implicar\'ia un gran riesgo, puesto que producir\.ia un mayor esfuerzo en la creaci\'on de funciones propias del proyecto, lo cual puede retrasar el desarrollo de manera exponencial si no se hace adecuadamente.

\item{Compleja interfaz gr\'afica para el usuario.}

Hevelius entre otros objetivos busca ser una herramienta amistosa para los operadores de telescopio, por lo tanto el tener una interfaz muy engorrosa volver\'ia la utilizaci\'on del proyecto muy compleja y poco entendible, para esto se deben utilizar herramientas y t\'ecnicas que permitan un desarrollo de interfaz amigable.

\item{Pocas pruebas realizadas al proyecto.}

Una herramienta importante en el desarrollo de software es la fase de pruebas, una vaga fase de pruebas al proyecto puede significar errores que no son identificados en la etapa correcta y que conlleva a errores en futuras etapas lo que implica un cambio en la estructura del proyecto, los cuales son mas costosos a medida de que el proyecto avanza.
\end{itemize}
\end{itemize}
\begin{tabular}{||l | c | r||}
\hline
Nombre del Riesgo & Ocurrencia & Impacto \\ 
\hline
Poca escalabilidad de Hevelius con respecto a telescopios profesionales. & Muy Alto & Muy Alto\\
\hline
Ausencia de alg\'un integrante del proyecto & Baja & Muy Alto\\
\hline
Incumplimiento con fechas & Media & Alto\\
\hline
Telescopio amateur no disponible & Baja & Muy Alto\\
\hline
Estimaci\'on err\'onea del proyecto & Media & Muy Alto\\
\hline
Falta de conocimiento en cuanto al \'area de desarrollo & Media & Alto\\
\hline
No cumplimiento con los est\'andares del proyecto ALMA-CONICYT & Baja & Alto\\
\hline
Mala elecci\'on en cuanto a lenguajes de programaci\'on & Baja & Alto\\
\hline
Compleja interfaz grafica para el usuario & Alta & Muy Alto\\
\hline
Pocas pruebas realizadas al proyecto & Media & Alto\\
\hline
\end{tabular}

Para priorizar los riesgos es necesario evaluar tanto su ocurrencia como el impacto que provocar\'ia en el proyecto.
Se debe crear un consenso en el equipo de trabajo para poder tomar la decisi\'on correcta.

\begin{enumerate}
	\item Poca escalabilidad de Hevelius con respecto a telescopios profesionales.
	\item Compleja interfaz grafica para el usuario.
	\item Estimaci\'on del proyecto err\'onea.
	\item Falta de conocimiento en cuanto al \'area de desarrollo.
	\item Pocas pruebas realizadas al proyecto.
	\item No cumplimiento con las fechas.
	\item Mala elecci\'on en cuanto a lenguajes de programaci\'on.
	\item Telescopio amateur no disponible.
	\item Ausencia de alg\'un integrante del proyecto.
	\item No cumplimiento con los est\'andares del proyecto ALMA-CONICYT.
\end{enumerate}


\subsection{Preparaci\'on para control de riesgos}

%%%FALTAN LAS CARTAS PERO SON DEMASIADA GESTION PARA UNA SOLA PERSONA X_X %%%
%%%%% MARINA 5 y ESTEBAN 5


\newpage
\section{Implementaci\'on (Entrega y Operaci\'on)} %%% MARINA
\subsection{Plan de operaci\'on del sistema}
Los componentes computacionales m\'inimos requeridos por Hevelius para su 
operaci\'on consisten en un computador con Sistema Operativo Linux Fedora Core 
y Software ACS 6.0, no se restringe s\'olo a la utilizaci\'on de esta versi\'on, puede 
utilizar otras, pero con las distintas versiones puede ser que existan variaciones 
que impliquen modificaciones al c\'odigo fuente, pero se deja 
establecido que en la versi\'on 6.0 funcionar\'a correctamente, de acuerdo 
a los requerimientos del cliente.

El equipo en el cual se implemente Hevelius tambi\'en debe poseer acceso 
a Internet y sin olvidar el acceso al telescopio que se desea operar. Sobre 
los requerimientos m\'inimos de hardware a\'un no est\'an definidos.

Hevelius se desarrollar\'a sobre la plataforma Linux Fedora Core y Software ACS 6.0 como 
ya se hab\'ia especificado y con el telescopio NEXSTAR 4 SE y a\~nadido 
a \'este una c\'amara CCD para la obtenci\'on de im\'agenes.

Como Hevelius es s\'olo el primer paso para el desarrollo completo de un 
software de control gen\'erico para telescopios, es muy importante la 
comprensi\'on del c\'odigo entregado, debidamente comentado, como requerimiento 
del cliente en ingl\'es y, de la misma forma, informar los avances en twiki 
de ACS-UTFSM Group, para que posteriormente pueda ser modificado de acuerdo 
a requerimientos futuros.

\subsection{Plan de implementaci\'on (entrega)}
Una vez finalizado el desarrollo del software, el proceso de entrega debe 
consistir de dos fases.

\begin{enumerate}
	\item {\bf{Entrega del programa y c\'odigo.}}\\
Como ya se ha mencionado anteriormente Hevelius es un paso a la construcci\'on 
de un software gen\'erico, es por ello la importancia del c\'odigo, puesto que 
es la base para que posteriormente se siga desarrollando en este tema, por 
estas razones se entrega el c\'odigo debidamente ordenado, organizado y 
comentado en ingl\'es, por ser nuestro cliente de car\'acter internacional.

En lo que se refiere al programa en s\'i, no se puede hacer una capacitaci\'on 
a quienes usar\'an este software, ya que no son personas espec\'ificas. Pero al 
finalizar el desarrollo de Hevelius se tratar\'a que astr\'onomos prueben el 
funcionamiento del software. Es por esto, que para aquellos que deban tratar 
con Hevelius, existe una documentaci\'on en la cual se detalla los componentes 
y la utilizaci\'on de ellos, esta documentaci\'on ser\'a especificada en la 
siguiente fase.\\

	\item{\bf{Documentaci\'on.}}\\
Como Hevelius est\'a siendo creado para personas especializadas en el tema de 
la astronom\'ia, se les entrega documentaci\'on detallada del software, ya que 
no existe una instancia directa en donde se pueda preguntar acerca de su 
funcionamiento, donde el \'unico contacto podr\'ia ser mediante correo 
electr\'onico, puesto que el ambiente en que trabaja Hevelius es el de los 
observatorios y, por lo tanto, el trato directo se hace m\'as complicado.

La especificaci\'on de la documentaci\'on consiste en las siguientes partes:

\begin{itemize}
	\item {\bf{Explicaci\'on de la interfaz:}} Esta consiste en la 
explicaci\'on de d\'onde se encuentra ubicado cada uno de los componentes que 
tiene implementado Hevelius.
	\item{ \bf{Componentes Implementados:}} En una secci\'on se especifica 
qu\'e hace cada uno de sus componentes y c\'omo es el funcionamiento de ellos, 
qu\'e par\'ametros recibe, etc.

Toda la documentaci\'on debe ser desarrollada en ingl\'es.
\end{itemize}
\end{enumerate}

\subsection{Plan de mantenci\'on}
Como ya se ha mencionado anteriormente Hevelius esta implementado mayormente 
para observatorios, los cuales se encuentran en distintas partes del mundo y 
los usuarios del programa pueden acceder desde donde prefieran para manipular 
los telescopios, por lo que nos es imposible dar mantenimiento presencial a 
todos los usuarios.\\

Puede existir una asistencia remota, principalmente a trav\'es de correo 
electr\'onico para tratar de resolver cualquier tipo de problema que pueda 
existir.



\newpage
\section{Planificaci\'on de Actividades} %%% CARLOS
\subsection{Work Breakdown Structure (WBS)}



\subsection{Carta Gantt}



\subsection{Resumen de Compromisos}




\end{document}
