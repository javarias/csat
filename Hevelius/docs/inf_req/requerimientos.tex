\documentclass[letterpaper,spanish,10pt]{article}
\usepackage[latin1]{inputenc}    % Agregar y acentos
\usepackage{babel}               % Soporte multilenguajes
\usepackage{avant}               % Tipo de fuente
%\usepackage{fancyheadings}       %% Topes y pies de pï¿œina
\usepackage[dvips]{graphicx}     % Inclusion de imagenes .eps
\usepackage{url}                 % Agregar Links soporte de ~
\usepackage{verbatim}
\usepackage{geometry}
\usepackage{url}
\usepackage{amsfonts}
\usepackage{amssymb}
%\usepackage{txfonts}
%\usepackage{emphoff}
%\usepackage{pxfonts}
%\usepackage{fancybox}
\usepackage{latexsym}
%\usepackage{fancyvrb}
\usepackage{graphicx}
%\usepackage{wasysym}
%\renewcommand{\baselinestretch}{1.5}
\parskip=7mm
\pagestyle{myheadings}
\geometry{tmargin=2.5cm, bmargin=2.5cm, lmargin=2.5cm, rmargin=2.5cm}
\markright{\hrulefill Proyecto Hevelius $\; \;$}


%opening
\title{{\Huge \bf Proyecto Hevelius} \\ {\Large Empresa DevNull} \\ {\small Requerimientos}}

\author{
{\bf Carlos Guajardo Miranda} \\ Jefe de Proyecto \\ \url{cguajard@alumnos.inf.utfsm.cl} \\ cel. 09-95046118 
\and
{\bf Marina Pilar Daza} \\ Miembro del Equipo \\ \url{mpilar@alumnos.inf.utfsm.cl} \\ cel. 09-84085407
\and
{\bf Esteban Espinoza Mart\'inez} \\ Miembro del Equipo \\ \url{eespinoz@alumnos.inf.utfsm.cl} \\ cel. 09-85596939
\and
{\bf Tom\'as Staig Fern\'andez} \\ Miembro del Equipo \\ \url{tstaig@alumnos.inf.utfsm.cl} \\ cel. 09-97615666
}


\date{13 de abril de 2007}


\begin{document}

\maketitle
%\tableofcontents


%\input{parte1}
%\input{parte2}
\newpage

\section{Resumen Proyecto}
El proyecto Hevelius, a cargo de la pre-Empresa DevNull, pretende, como objetivo principal, crear un software de control de telescopios, que sirva de impulso hacia la creaci\'on de un control gen\'erico de telespios. Hevelius, debe poseer flexibilidad en su estructura como software, de manera que permita la futura implementaci\'on de funcionalidades extras y el reemplazo de otras.

\begin{center}
\it{``Control de telescopio amateur bajo interfaz de control profesional''}
\end{center}


\section{Lista de Requerimientos}
\subsection{Requerimientos Funcionales Propias de la Aplicaci\'on}
\begin{itemize}
	\item Ajustar posici\'on del telescopio bajo sistema de coordenadas ecuatoriales.
	\item Mover el telescopio a la velocidad de la hora sideral.
	\item Impedir observaci\'on a lugares donde exista Luminosidad Lunar.
	\item Mostrar un modelo visual del estado f\'isico del telescopio.
	\item Ajustar manualmente el telescopio.
%	\item Conectar software a alg\'un cat\'alogo de estrellas.
	\item Detener de forma inmediata el telescopio en caso de emergencia.
%	\item Adquirir datos de la observaci\'on (videos, im\'agenes, etc.)
\end{itemize}


\subsection{Requerimientos Funcionales Gen\'ericos}
\begin{itemize}
	\item Guardar coordenadas de observaciones realizadas.
	\item Controlar acceso a la aplicaci\'on (sesiones).
\end{itemize}


\subsection{Requerimientos Funcionales Distintivos}
\begin{itemize}
	\item Crear software bajo plataforma ACS.
\end{itemize}

\subsection{Requerimientos No Funcionales Distintivos}
\begin{itemize}
	\item Permitir conexion a sistema diferente de Pointing Autom\'atico.
	\item Documentar, crear manuales y/o informes t\'ecnicos con formato est\'andar de ALMA.
\end{itemize}


\newpage


\section{Especificaciones del Software}
La plataforma en la que se desarrollar\'a el proyecto se basa en los requerimientos de nuestro cliente, la cual consiste en el Sistema Operativo Linux y el software ACS 6.0.

\subsection{Especificaci\'on de Hardware}
Los equipos con los cuales se desarrollar\'a nuestro proyecto poseen de las siguientes caracter\'isticas:

\begin{itemize}
	\item Procesador Pentium IV 3.0 Ghz de 32 bits.
	\item 1Gb RAM
	\item Telescopio NEXSTAR 4 SE que posee computador manual de 16 car\'acteres con pantalla LCD y 19 botones. Algunas de sus caracter\'isticas: apertura de 102 mm, largo focal de 1325 mm, radio focal de 12.99 mm, entre otras cosas.
	\item Se a\~nadir\'a al telescopio un CCS para poder obtener im\'agenes.
\end{itemize}

El riesgo que se corre con respecto a la adquicici\'on del hardware es la obtenci\'on del telescopio, puesto que tanto el resto del equipo a utilizar y la plataforma de software ya est\'an disponibles.
Sin embargo, a la fecha de la creaci\'on de este informe, se inform\'o que el Telescopio ser\'a adquirido dentro de esta semana (9 al 13 de abril), por lo que el riesgo de hardware se reduce considerablemente.

\newpage


\section{Presentaci\'on Cliente y Usuario}
Hevelius se enmarca en un proyecto de mayor magnitud, llamado ``Software development for ALMA-CONICYT: Building up expertise to meet ALMA-CONICYT software requirements within a Chilean University''. Los representantes de dicho proyecto son Mauricio Araya y Horst Von Brand.

{\bf Mauricio Alejandro Araya L\'opez}, estudiante de Ingenier\'ia Civil Inform\'atica y Mag\'ister en Inform\'atica en la Universidad T\'ecnica Federico Santa Mar\'ia. Se ha desempe\~nado como Team Leader y Partnership del ACS-UTFSM, donde colabor\'o en el ALMA Common Software, liderado por European Southern Observatory en diversas aplicaciones sobre ACS y en el desarrollo del mismo framework ACS. 

{\bf Horst von Brand}, PhD en Computer Science de la Louisiana State University. Actualmente se desempe\~na como profesor del Departamento de Inform\'atica de la Universidad T\'ecnica Federico Santa Mar\'ia, adem\'as de ser Jefe del Proyecto ``Software development for ALMA-CONICYT: Building up expertise to meet ALMA-CONICYT software requirements within a Chilean University''.



\subsection{Contacto}
A trav\'es de la p\'agina web del proyecto \url{www.acs.inf.utfsm.cl}
A Mauricio Araya, en el Laboraotorio de Sistemas Distribuidos, Departamento de Inform\'atica, Edificio F2, Piso 1, UTFSM; o mediante correo electr\'onico \url{maray@inf.utfsm.cl}.
A Horst von Brand, a trav\'es de la Secretar\'ia del Departamento de Inform\'atica, Edificio F2, Piso 1, UTFSM; o mediante su correo electr\'onico \url{vonbrand@inf.utfsm.cl}.


\newpage


\section{Anexo: Detalle de los Requerimientos}

\subsection{Requerimientos Funcionales Propias de la Aplicaci\'on}

\subsubsection{Ajustar posici\'on del telescopio bajo sistema de coordenadas ecuatoriales.}
Es necesario ajustar el telescopio a una posici\'on fija, esto debe hacerse respecto al sistema de coordenadas ecuatoriales R.A.DEC., el cual permite conocer la ubicaci\'on de la estrella por periodos, es decir, se puede considerar que la estrella no se mueve en periodos cortos de tiempo. Pero todav\'ia queda el problema de que la tierra se mueve a medida que pasa el tiempo. Para esto, se utiliza la hora sideral, la cual considera estos movimientos, de manera que a una misma hora en d\'ias distintos, durante alguno de estos periodos se puede ver la estrella en el mismo lugar. El presetting no puede demorar mucho, por lo que los movimientos del telescopio tienen que ser relativamente r\'apidos \footnote{M\'as Informaci\'on: \url{http://es.wikipedia.org/wiki/Coordenadas_ecuatoriales}}.

\subsubsection{Mover el telescopio a la velocidad de la hora sideral.}
El sistema debe dar la posibilidad de tracking, de manera que el telescopio pueda seguir el punto que se est\'a observando a trav\'es del tiempo, a pesar de los movimientos de la tierra. Para esto, se debe implementar un sistema distinto al presetting, en el sentido que este tiene que ser m\'as sutil, ya que los movimientos son mucho m\'as peque\~nos, y no queremos que el telescopio se mueva mucho para no afectar a los estudios que se puedan estar haciendo.

\subsubsection{Impedir observaci\'on a lugares donde exista Luminosidad Lunar.}
Al observar las estrellas en presencia de la luna, uno se percata que existe una disminuci\'on en la calidad de la observaci\'on debido a la luminosidad de la luna y a su volumen. El software Hevelius debe advertir al controlador del telescopio que este se dirige a un punto en donde la observaci\'on no ser\'a de calidad, por lo que se cuidar\'an tanto el tiempo de observaci\'on como los lentes de los telescopios sensibles a mucha luminosidad.

\subsubsection{Mostrar un modelo visual del estado f\'isico del telescopio.}
Un operador de telescopio puede estar situado: en el telescopio mismo, en un centro de control, o en cualquier otro lugar, por lo que no siempre tendr\'a la oportunidad de observar cu\'al es el movimiento real del telescopio que controla. Hevelius debe simular el movimiento del telescopio para que el operador, est\'e donde est\'e, pueda tener una visi\'on del comportamiento de este y as\'i permitir diagnosticar alguna falla en la observaci\'on.

\subsubsection{Ajustar manualmente la posici\'on del telescopio.}
Al realizar las tareas de presetting y tracking existe la posibilidad de que no se vea lo que se supone, es decir, que exista un ligero corrimiento de lo que se debiera ver, por diversos motivos, como son las caracter\'isticas de material del telescopio, ya que en algunas posiciones este puede doblarse por acci\'on de su propio peso, generando una peque\~na diferencia con lo esperado. Es por esto que se requiere de un sistema de pointing manual, que permita al operador del telescopio realizar movimientos manuales para ajustar el telescopio. Cabe destacar, que este sistema de pointing debe ir por debajo del presetting y del tracking, afectando a lo que se ve realmente, pero sin cambiar las coordenadas y horas con las que ellos trabajan.

%\subsubsection{Conectar software a alg\'un cat\'alogo de estrellas.}

\subsubsection{Detener de forma inmediata el telescopio en caso de emergencia.}
En general, los telescopios son aparatos delicados a condiciones clim\'aticas, de luz y muchos otros factores. Es por esto que se debe contar con una opci\'on para proteger al telescopio de estas situaciones. Esta opci\'on es un bot\'on de p\'anico que debe hacer que el telescopio quede protegido de manera ``instant\'anea'' y, debe lograrlo, considerando que trabaja sobre la plataforma ACS de manera distribuida, es decir, puede estar siendo operado de cualquier parte del mundo. Se debe asegurar que a\'un en estas situaciones, el bot\'on de p\'anico funcione r\'apidamente.

%\subsubsection{Adquirir datos de la observaci\'on (videos, im\'agenes, etc.)}


\subsection{Requerimientos Funcionales Gen\'ericos}
\subsubsection{Guardar coordenadas de observaciones realizadas.}
Cada vez que se utiliza Hevelius se realiza una serie de investigaciones u observaciones que son de gran importancia para los que lo realizan, es por ello que Hevelius guardar\'a un registro de todas las coordenadas de observaci\'on que se realicen durante una sesi\'on, generando un reporte con la informaci\'on obtenida.

\subsubsection{Controlar acceso a la aplicaci\'on (sesiones).}
Para tener un control de la gente que usar\'a Hevelius, se implementar\'a un sistema de sesiones, en donde cada usuario debidamente registrado, ingresar\'a al software para realizar sus trabajos.


\subsection{Requerimientos Funcionales Distintivos}
\subsubsection{Crear software bajo plataforma ACS.}
La plataforma ACS \footnote{M\'as Informaci\'on: \url{http://www.eso.org/~almamgr/AlmaAcs/index.html}} (Alma Common Software) es un framework que facilita el uso de sistemas distribuidos. Se caracteriza por su modelo Componente-Container, el cual permite utilizar aplicaciones desarrolladas en otro sitio como si estuvieran donde uno desarrolla. Hevelius debe utilizar aplicaciones ya creadas (para no replicar c\'odigo), crear aplicaciones para el resto del mundo y utilizar todo el sistema en s\'i para contruir el software.


\subsection{Requerimientos No Funcionales Distintivos}
\subsubsection{Permitir conexion a sistema diferente de Pointing Autom\'atico.}
El pointing, posicionamiento del telescopio en alguna coordenada, puede ser desarrollado de diversas formas, seg\'un el telescopio a utilizar. Hevelius ser\'a capaz de poder ajustarse a nuevos sistemas de pointing autom\'atico, lo cual permite a Hevelius ser un software modular.

\subsubsection{Documentar, crear manuales e/o informes t\'ecnicos con formato est\'andar de ALMA.}
El proyecto Hevelius al enmarcarse en el proyecto de ALMA y Conicyt ``Software development for ALMA-CONICYT: Building up expertise to meet ALMA-CONICYT software requirements within a Chilean University'', debe cumplir ciertos est\'andares, como son el formato y lenguajes de programaci\'on. 
Toda la documentaci\'on del proyecto y comentarios en el c\'odigo fuente estar\'an realizados en Ingl\'es, puesto que el proyecto tiene un car\'acter internacional.
Otro est\'andar a seguir, son los tipos de lenguajes permitidos, puesto que el proyecto funciona con el framework ACS 6.0, Hevelius puede utilizar s\'olo los lenguajes Java, Phyton, C y C++.

\end{document}
