\documentclass[letterpaper,titlepage,spanish,10pt]{article}
\usepackage[latin1]{inputenc}    % Agregar y acentos
\usepackage{babel}               % Soporte multilenguajes
\usepackage{avant}               % Tipo de fuente
%\usepackage{fancyheadings}      % Topes y pies de p'agina
\usepackage[dvips]{graphicx}     % Inclusion de imagenes .eps
\usepackage{url}                 % Agregar Links soporte de ~
\usepackage{verbatim}
\usepackage{geometry}
\usepackage{url}
\usepackage{amsfonts}
\usepackage{amssymb}
%\usepackage{txfonts}
%\usepackage{emphoff}
%\usepackage{pxfonts}
%\usepackage{fancybox}
\usepackage{latexsym}
%\usepackage{fancyvrb}
\usepackage{graphicx}
%\usepackage{wasysym}
%\renewcommand{\baselinestretch}{1.5}
\parskip=7mm
\pagestyle{myheadings}
\geometry{tmargin=4cm, bmargin=4cm, lmargin=2.5cm, rmargin=2.5cm}
\markright{\hrulefill Plan de Negocio - Software Hevelius $\; \;$}


%opening
\title{{\Huge \bf Plan de Negocio} \\ {\Large Software Hevelius} \\ {\small Empresa DevNull}}

\author{
{\bf Carlos Guajardo Miranda} \\ \url{cguajard@alumnos.inf.utfsm.cl}
\and
{\bf Marina Pilar Daza} \\ \url{mpilar@alumnos.inf.utfsm.cl}
\and
{\bf Esteban Espinoza Mart\'inez} \\ \url{eespinoz@alumnos.inf.utfsm.cl}
\and
{\bf Tom\'as Staig Fern\'andez} \\ \url{tstaig@alumnos.inf.utfsm.cl}
}


\date{11 de octubre de 2007}


\begin{document}

% Portada
\maketitle
\newpage

% Indices
\tableofcontents{}
\newpage

\section{Resumen Ejecutivo}
%Descripci�n: Documento de presentaci�n que motiva a leer el resto del documento por el inter�s que despierta.



\newpage
\section{El Producto: \textit{Hevelius}}
%    *  Presentaci�n breve pero completa del producto
%    * �Qu� problema soluciona?
%    * �A qu� cliente responde y c�mo es el tipo de usuario?
%    * �Qu� es lo innovador que lo diferencia de la competencia (distintivo)?
%    * Hacer una breve referencia a productos similares existentes en el mercado global (mundial).



\newpage
\section{El Mercado}
%    *  �C�mo es el cliente real (superficie)?
%    * �A qu� segmento de mercado est� orientado?
%    * �Qu� posibilidades de evoluci�n de mercado existe (mercado potencial)?
%    * �Cu�l es el tama�o estimado? (Cuantificar).
%    * �Cu�l es mi competencia actual y potencial?
%    * Posibilidad del servicio como valor agregado.
%    * �De qu� forma se piensa penetrar, difundir en el mercado, mediante estrategias de marketing?

El cliente de Hevelius es un grupo de investigaci\'on astron\'omica que trabaja en el desarrollo de TCS 
(Sistemas de Control de Telescopios, por sus siglas en ingl\'es).
\\

Hevelius est\'a orientado a gente que trabaja con telescopios, ya sea en su proceso de operaci\'on o
en el estudio de las im\'agenes que el software provee.
\\

La gracia de Hevelius es que facilita el control de los telescopios, por lo que si la experiencia de 
nuestro cliente es exitosa, podemos tener a organizaciones completas (las cuales tienen diversos observatorios)
como futuros clientes y usuarios de nuestro software a nivel mundial.
\\

Si estimamos que un observatorio cuenta con un m\'inimo de 3 telescopios que usar\'an nuestro software 
y decimos que s\'olo en chile existen alrededor de 15 observatorios, son 45 unidades, vendibles, del software en Chile.
\\

Actualmente, no existe un software de las caracter\'isticas de Hevelius, por lo que no tenemos competencia 
en cuanto a la generalidad del uso del software. Sin embargo, el hecho de tener software creado espec\'ificamente
para cada telescopio es contar con una competencia fuerte que debe ser desplazada.

Existen reportes de que en Europa hay un grupo que desarrolla un software con el mismo fundamento que 
Hevelius, control gen\'erico de los telescopios, por lo que este futuro software ser\'a una competencia a tener
en cuenta.
\\

Para darle un valor agregado a nuestro producto, se puede ofrecer mantenci\'on y/o actualizaciones
de software a los clientes, lo que implica 2 l\'ineas de desarrollo: una que se dedica a corregir, mejorar 
el software comprado; y otra, que se dedica a crear una nueva versi\'on de Hevelius con nuevas caracter\'isticas
y que permite un nuevo negocio al ofrecer un software m\'as moderno y actualizado que el que se compr\'o.
\\

La mejor manera de entrar en el mercado cient\'fico-astron\'omico es demostrando con pruebas que el software
funciona realmente, por lo que es fundamental que nuestro cliente quede satisfecho con nuestro producto y as\'i 
se pueda usar esta experiencia como prueba fiel de \'exito.


\newpage
\section{Precio del Producto}
%    *  �Cu�l es el costo real de los desarrolladores (horas de trabajo por tipo de trabajo)?
%    * �Cu�l es la inversi�n y cu�nto es la inversi�n en Hardware y Software necesario?
%    * Identificar cu�les son los costos que existen antes de la puesta en marcha del producto. (arriendo, tel�fono, traslados, etc.).
%    * Identificar cu�les son los costos de explotaci�n (de venta, mantenedores del producto, aparatos, etc.)
%    * Identificar cu�les son los impuestos y los gastos financieros que existen.
%    * Investigue y verifique cu�l es el precio actual de mercado de los productos similares.
%    * Determine cu�l es el precio estimado del producto (estimando n�mero de productos a vender y la utilidad esperada).





\newpage
\section{Los Emprendedores}
%    *  Perfil de la empresa
%    * Curr�culum de los integrantes (una p�gina cada uno).






\end{document}
