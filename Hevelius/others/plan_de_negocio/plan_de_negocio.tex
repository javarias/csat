\documentclass[letterpaper,titlepage,spanish,10pt]{article}
\usepackage[latin1]{inputenc}    % Agregar y acentos
\usepackage{babel}               % Soporte multilenguajes
\usepackage{avant}               % Tipo de fuente
%\usepackage{fancyheadings}      % Topes y pies de p'agina
\usepackage[dvips]{graphicx}     % Inclusion de imagenes .eps
\usepackage{url}                 % Agregar Links soporte de ~
\usepackage{verbatim}
\usepackage{geometry}
\usepackage{url}
\usepackage{amsfonts}
\usepackage{amssymb}
%\usepackage{txfonts}
%\usepackage{emphoff}
%\usepackage{pxfonts}
%\usepackage{fancybox}
\usepackage{latexsym}
%\usepackage{fancyvrb}
\usepackage{graphicx}
%\usepackage{wasysym}
%\renewcommand{\baselinestretch}{1.5}
\parskip=7mm
\pagestyle{myheadings}
\geometry{tmargin=4cm, bmargin=4cm, lmargin=2.5cm, rmargin=2.5cm}
\markright{\hrulefill Plan de Negocio - Software Hevelius $\; \;$}


%opening
\title{{\Huge \bf Plan de Negocio} \\ {\Large Software Hevelius} \\ {\small Empresa DevNull}}

\author{
{\bf Carlos Guajardo Miranda} \\ \url{cguajard@alumnos.inf.utfsm.cl}
\and
{\bf Marina Pilar Daza} \\ \url{mpilar@alumnos.inf.utfsm.cl}
\and
{\bf Esteban Espinoza Mart\'inez} \\ \url{eespinoz@alumnos.inf.utfsm.cl}
\and
{\bf Tom\'as Staig Fern\'andez} \\ \url{tstaig@alumnos.inf.utfsm.cl}
}


\date{11 de octubre de 2007}


\begin{document}

% Portada
\maketitle
\newpage

% Indices
\tableofcontents{}
\newpage

\section{Resumen Ejecutivo}
%Descripci�n: Documento de presentaci�n que motiva a leer el resto del documento por el inter�s que despierta.



\newpage
\section{El Producto: \textit{Hevelius}}
%    *  Presentaci�n breve pero completa del producto
%    * �Qu� problema soluciona?
%    * �A qu� cliente responde y c�mo es el tipo de usuario?
%    * �Qu� es lo innovador que lo diferencia de la competencia (distintivo)?
%    * Hacer una breve referencia a productos similares existentes en el mercado global (mundial).



\newpage
\section{El Mercado}
%    *  �C�mo es el cliente real (superficie)?
%    * �A qu� segmento de mercado est� orientado?
%    * �Qu� posibilidades de evoluci�n de mercado existe (mercado potencial)?
%    * �Cu�l es el tama�o estimado? (Cuantificar).
%    * �Cu�l es mi competencia actual y potencial?
%    * Posibilidad del servicio como valor agregado.
%    * �De qu� forma se piensa penetrar, difundir en el mercado, mediante estrategias de marketing?

El cliente de Hevelius es un grupo de investigaci\'on astron\'omica que trabaja en el desarrollo de TCS 
(Sistemas de Control de Telescopios, por sus siglas en ingl\'es).
\\

Hevelius est\'a orientado a gente que trabaja con telescopios, ya sea en su proceso de operaci\'on o
en el estudio de las im\'agenes que el software provee.
\\




\newpage
\section{Precio del Producto}
%    *  �Cu�l es el costo real de los desarrolladores (horas de trabajo por tipo de trabajo)?
%    * �Cu�l es la inversi�n y cu�nto es la inversi�n en Hardware y Software necesario?
%    * Identificar cu�les son los costos que existen antes de la puesta en marcha del producto. (arriendo, tel�fono, traslados, etc.).
%    * Identificar cu�les son los costos de explotaci�n (de venta, mantenedores del producto, aparatos, etc.)
%    * Identificar cu�les son los impuestos y los gastos financieros que existen.
%    * Investigue y verifique cu�l es el precio actual de mercado de los productos similares.
%    * Determine cu�l es el precio estimado del producto (estimando n�mero de productos a vender y la utilidad esperada).





\newpage
\section{Los Emprendedores}
%    *  Perfil de la empresa
%    * Curr�culum de los integrantes (una p�gina cada uno).






\end{document}
